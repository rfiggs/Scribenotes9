\documentclass[11pt]{article}
\usepackage{latexsym}
\usepackage{amsmath}
\usepackage{algorithm}
\usepackage{algorithmic}
\usepackage{amssymb}
\usepackage{amsthm}
\usepackage{epsfig}
\usepackage{graphicx}

\newcommand{\handout}[5]{
  \noindent
  \begin{center}
  \framebox{
    \vbox{
      \hbox to 5.78in { {\bf ICS 443: Parallel Algorithms} \hfill #2 }
      \vspace{4mm}
      \hbox to 5.78in { {\Large \hfill #5  \hfill} }
      \vspace{2mm}
      \hbox to 5.78in { {\em #3 \hfill #4} }
    }
  }
  \end{center}
  \vspace*{4mm}
}
\newcommand{\lecture}[4]{\handout{#1}{#2}{#3}{Scribes: #4}{Lecture #1}}

\newtheorem{theorem}{Theorem}
\newtheorem{corollary}[theorem]{Corollary}
\newtheorem{lemma}[theorem]{Lemma}
\newtheorem{observation}[theorem]{Observation}
\newtheorem{proposition}[theorem]{Proposition}
\newtheorem{definition}[theorem]{Definition}
\newtheorem{claim}[theorem]{Claim}
\newtheorem{fact}[theorem]{Fact}
\newtheorem{assumption}[theorem]{Assumption}

% 1-inch margins, from fullpage.sty by H.Partl, Version 2, Dec. 15, 1988.
\topmargin 0pt
\advance \topmargin by -\headheight
\advance \topmargin by -\headsep
\textheight 8.9in
\oddsidemargin 0pt
\evensidemargin \oddsidemargin
\marginparwidth 0.5in
\textwidth 6.5in

\parindent 0in
\parskip 1.5ex
%\renewcommand{\baselinestretch}{1.25}

\begin{document}

\lecture{9 --- September 25, 2017}{Fall 2017}{Prof.\ Nodari Sitchinava}{Isaac DeMello, Brandon Doan, Robert Figgs}

\section{Overview}

In the last lecture we detailed an algorithm that finds the minimum of an array in O$(\log_2(\log_2(n)))$ time, and with O$(n(\log_2(\log_2(n)))$ work.

In this lecture we define a work-efficient algorithm that finds the minimum in O$(\log_2(\log_2(n)))$ and with O$(n)$ work. We also begin to look at searching algorithms.  \ldots.

\section{Work Efficient Min}

We begin by breaking down the algorithm into parts of size k, where k = O$(\log_2(\log_2(n)))$
\subsection{WE-CRCW-min}
In order to break down the size of the array and apply the previously defined CRCW-min algorithm covered in lecture 9. We use the following algorithm:

\begin{algorithm}
\caption{WE-CRCW-min$(A,n)$}
\begin{algorithmic} 
\STATE k = $\lceil\log_2(\log_2(n))\rceil$
\STATE n' = $\lceil n/k \rceil$
\STATE B = new array of size n'
\FOR {$i=1$ to $n'$ in parallel}
	\STATE {B[i]=A[(i-1)k+1]}
	\FOR {$j = (i-1)k+2$ to $ik$ in parallel}
    \STATE $B[i] = min(B[i], A[j])$
	\ENDFOR
\ENDFOR
\RETURN CRCW-min(B, 1, n')
\end{algorithmic}
\end{algorithm}


\subsubsection{WE-CRCW-min Algorithmic Analysis}
The first 3 lines of the algorithm can be done in constant time, this means that the runtime of the algorithm is dominated by the for-loop. 



\paragraph{Work Analysis}: Work of this algorithm can be shown as follows:


$\sum_{i=1}^{n} \sum_{j=(i-1)k+2}^{ik}O(1)$ = 
$\theta(n'*k)$ = 
$\theta((n/k) * k)$ = $\theta(n)$	

\section{Searching}
In this part of the lecture we look at how to both efficiently and inefficiently search through an array for a specified element.

\subsection{Inefficient Search}
\begin{algorithm}
\caption{Searching$(A, n, x)$}
\begin{algorithmic}
\STATE answer = -1
\FOR {$i=1$ to $n$ in parallel}
\STATE if A[mid] ==x
	\STATE answer = i 
\ENDFOR
\RETURN answer
\end{algorithmic}
\end{algorithm}

\subsection{Recursive Search}
While searching a sorted input, it is impossible to improve the run time of the search. The algorithm is as follows:

\begin{algorithm}
\caption{Binary-Search($A, left, r, x$)}
\begin{algorithmic} 
\STATE mid = $\lfloor (l + r)/2 \rfloor$
\IF {r < l}
	\RETURN -1
\ENDIF
\IF{$A[mid] == x$}
	\RETURN x
\ENDIF
\IF {x > A[mid]}
	\RETURN return Binary-Search($A, l, mid+1, r, x$)
\ELSE
	\RETURN Binary-Search($A, l, mid-1, x$)
\ENDIF
\end{algorithmic}
\end{algorithm}


\end{document}
                                       
